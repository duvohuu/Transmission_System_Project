\chapter{THIẾT KẾ BỘ TRUYỀN BÁNH RĂNG CÔN (HỞ)}
    \section*{4.0. THÔNG SỐ KỸ THUẬT BỘ TRUYỀN BÁNH RĂNG CÔN}
        \begin{itemize}
            \item Công suất trục dẫn: $P_{III} = 2.72 (kW)$
            \item Công suất trục bị dẫn: $P_{IV} = 2.5 (kW)$
            \item Momen xoắn trục dẫn: $T_{III} = 360.78 (N.m)$
            \item Momen xoắn trục bị dẫn: $T_{IV} = 1989.58 (N.m)$
            \item Số vòng quay trục dẫn: $n_{III} = 72 (vg/ph)$
            \item Số vòng quay trục bị dẫn: $n_{IV} = 12 (vg/ph)$
            \item Tỷ số truyền:: $u_{34} = 6$
            \item Thời gian làm việc: $L_h = 300 \cdot 8 \cdot 2 \cdot 5 = 24000 (h)$
        \end{itemize}
    \section{Chọn vật liệu chế tạo bánh răng, phương pháp nhiệt luyện, cơ tính vật liệu}
        \subsection{Chọn vật liệu chế tạo bánh răng}
            \hspace*{0.6cm}Theo Bảng 6.2, tài liệu tham khảo \cite{tltk1} Chọn thép C45 được tôi cải thiện có độ rắn đạt $HB = 180 \div 350$. 
            \begin{itemize}
                \item Đối với bánh răng dẫn, ta chọn độ rắn trung bình ở cả mặt răng và lõi răng là $H_{1} = 350 HB$. 
                \item Đối với bánh răng bị dẫn, ta chọn độ rắn trung bình ở cả mặt răng và lõi răng là $H_{2} = 320 HB$.
            \end{itemize}
            \hspace*{0.6cm}Ở đây mặc dù ta dùng bộ truyền bánh răng côn là bộ truyền hở, nên ta tiến hành tính toán thiết kế bánh răng theo độ bền uốn
        \subsection{Phương pháp nhiệt luyện và cơ tính vật liệu}
            \hspace*{0.6cm}Theo bảng 6.13, giới hạn mỏi tiếp xúc và uốn các bánh răng xác định như sau: \\[0.2cm]
            \hspace*{0.6cm}Từ công thức \ref{eq:3.1}:
            \[
            \Rightarrow
            \begin{cases}
                \sigma_{Hlim1} = 2 \cdot 350 + 70 = 770 \, \mathrm{MPa} \\
                \sigma_{Hlim2} = 2 \cdot 320 + 70 = 710 \, \mathrm{MPa}
            \end{cases}
            \] 
            \hspace*{0.6cm}Từ công thức \ref{eq:3.2}:
            \[
            \Rightarrow
            \begin{cases}
                \sigma_{Flim1} = 1.8 \cdot 350 = 630 \, \mathrm{MPa}. \\
                \sigma_{Flim2} = 1.8 \cdot 320 = 576 \, \mathrm{MPa}.
            \end{cases}
            \]
            \begin{table}[h]
                \centering
                \begin{tabular}{|>{\centering\arraybackslash}m{4.8cm}|>{\centering\arraybackslash}m{3cm}|>{\centering\arraybackslash}m{3cm}|}
                    \hline
                    \diagbox{\textbf{Thông số}}{\textbf{Bánh răng}} & \textbf{Bánh dẫn} & \textbf{Bánh bị dẫn} \\ 
                    \hline
                    \textbf{Loại thép} & C45 & C45 \\
                    \hline
                    \textbf{Nhiệt luyện} & Tôi cải thiện & Tôi cải thiện \\
                    \hline
                    \textbf{Độ rắn} & $HB_1 = 350$ & $HB_1 = 320$ \\ 
                    \hline
                    \textbf{Giới hạn mỏi(MPA)} & $\sigma_{Hlim1} = 770$ & $\sigma_{Hlim2} = 710$ \\
                    \hline
                    \textbf{Giới hạn uốn(MPA)} & $\sigma_{Flim1} = 630$ & $\sigma_{Flim2} = 576$ \\
                    \hline
                \end{tabular}
                \caption{Chọn vật liệu bánh răng}
                \label{tab:gear_ratios}
            \end{table}
        \section{Xác định ứng suất tiếp xúc  $[\sigma_H]$ và ứng suất uốn cho phép $[\sigma_F]$}
            \subsection{Số chu kỳ làm việc}
                \begin{itemize}
                    \item Số chu kỳ thay đổi ứng suất tương đương: \\[0.2cm]
                        Vì bộ truyền chịu tải trọng tĩnh:
                        Từ công thức \ref{eq:3.3}:
                        \begin{align*}
                            N_{HE1} = N_{FE1} = 60 \cdot c \cdot n_{III} \cdot L_h = 60 \cdot 1 \cdot 72 \cdot 24000 = 10368 \cdot 10^4 \\
                            N_{HE2} = N_{FE2} = 60 \cdot c \cdot n_{IV} \cdot L_h = 60 \cdot 1 \cdot 12 \cdot 24000 = 1728 \cdot 10^4
                        \end{align*}    
                    \item Số chu kỳ thay đổi ứng suất cơ sở: \\[0.2cm]
                        Từ công thức \ref{eq:3.4}:
                        \begin{align*}
                            N_{HO1} = 30 \cdot H_{1}^{2.4} = 30 \cdot 350^{2.4} = 38272299.91 \\
                            N_{HO2} = 30 \cdot H_{2}^{2.4} = 30 \cdot 330^{2.4} = 33231864.66 \\
                        \end{align*}
                        \begin{equation*}
                            N_{FO_1} = N_{FO_2} = 4 \cdot 10^6
                        \end{equation*}
                \end{itemize}
            \subsection{Ứng suất tiếp xúc cho phép}
                \hspace*{0.6cm}Theo công thức \ref{eq:3.5}, ta có ứng suất tiếp xúc cho phép:
                \begin{equation*}
                    [\sigma_{H}] = \sigma_{0Hlim} \cdot \frac{K_{HL}}{s_{H}}
                \end{equation*} 
                \hspace*{0.6cm}Trong đó:
                \begin{itemize}
                    \item Hệ số tuổi thọ $s_H = 1.1$ tra bảng 6.2 tài liệu tham khảo \cite{gtctm}.\\[0.2cm]
                    \item Hệ số tuổi thọ xét đến thời gian phục vụ:
                        \begin{equation*}
                            K_{HL} = \sqrt[m_H]{\frac{N_{HO}}{N_{HE}}}
                        \end{equation*}
                        Trong đó $m_H = 6$ do độ rắn các mặt răng đều có $HB < 350$.\\[0.2cm]
                        \[ 
                        \Rightarrow
                        \begin{cases}
                            K_{HL1} = \sqrt[6]{\frac{38272299.91}{864 \cdot 10^5}} = 0.873 \Rightarrow K_{HL1} = 1 \\
                            K_{HL2} = \sqrt[6]{\frac{33231864.66}{1728 \cdot 10^4}} = 1.12
                        \end{cases}
                        \
                        \]
                \end{itemize}
                \hspace*{0.6cm}Từ công thức \ref{eq:3.5}:
                \[
                \Rightarrow
                \begin{cases}
                    [\sigma_{H1}] = 770 \cdot \frac{1}{1.1} = 700 \, \mathrm{MPa} \\
                    [\sigma_{H2}] = 710 \cdot \frac{1.12}{1.1} = 722.91 \, \mathrm{MPa}
                \end{cases}
                \] 
            \subsection{Ứng suất uốn cho phép}
                \hspace*{0.6cm}Theo công thức \ref{eq:3.7}, ứng suất tiếp xúc cho phép:
                \begin{equation*}
                    [\sigma_{F}] = \sigma_{0Flim} \cdot \frac{K_{FC} \cdot K_{FL}}{s_{F}}
                \end{equation*} 
                \hspace*{0.6cm}Trong đó:
                \begin{itemize}
                    \item Hệ số tuổi thọ $s_F = 1.75$ tra bảng 6.2 tài liệu tham khảo \cite{gtctm}.
                    \item Hệ số xét đến ảnh hưởng đặt tải $K_{FC} = 1$ do đặt tải trọng 1 bên, bộ truyền quay một chiều.
                    \item Hệ số tuổi thọ xét đến chế độ tải trọng: 
                        \begin{equation*}
                            K_{FL} = \sqrt[m_F]{\frac{N_{FO}}{N_{FE}}} 
                        \end{equation*}
                        Trong đó $m_F = 6$ do độ rắn các mặt răng đều có $HB < 350$.\\[0.2cm]
                        \[
                        \Rightarrow
                        \begin{cases}
                            K_{FL1} = \sqrt[6]{\frac{4 \cdot 10^6}{864 \cdot 10^5}} = 0.599 \Rightarrow K_{FL1} = 1\\
                            K_{FL2} = \sqrt[6]{\frac{4 \cdot 10^6}{1728 \cdot 10^4}} = 0.784 \Rightarrow K_{FL2} = 1
                        \end{cases}
                        \]
                \end{itemize}
                \hspace*{0.6cm}Từ công thức \ref{eq:3.7}:
                \[
                \Rightarrow
                \begin{cases}
                    [\sigma_{F1}] = 630 \cdot \frac{1 \cdot 1}{1.75} = 360 \, \mathrm{MPa} \\
                    [\sigma_{F2}] = 576 \cdot \frac{1 \cdot 1}{1.75} = 329.14 \, \mathrm{MPa}
                \end{cases}
                \] 
    \section{TÍNH TOÁN BÁNH RĂNG THEO ĐỘ UỐN}
        \subsection{Tính toán số răng bánh dẫn và bánh bị dẫn}
            \hspace*{0.6cm}Chọn số răng bánh dẫn $z_1 = 25$, khi đó số răng bánh bị dẫn $z_2 = 6 \cdot 25 = 150$.
        \subsection{Xác định lại chính xác tỉ số truyền u và xác định các góc mặt côn chia $\delta_1$ và $\delta_2$}
            \begin{itemize}
                \item Tính toán lại tỉ số truyền:
                    $u = \frac{z_2}{z_1} = 5 \Rightarrow \Delta u = 0\% < 4\%$ (nằm trong khoảng cho phép).
                \item Góc mặt côn chia $\delta_1$ và $\delta_2$ được xác định theo công thức 6.98 tài liệu tham khảo \cite{gtctm}:
                    \begin{equation}
                        \delta_1 + \delta_2 = 90^{\circ}
                        \label{eq:4.1}
                    \end{equation}
                    Ta có: $\delta_1 = \arctan{\frac{z_1}{z_2}} = \arctan{\frac{25}{150}} = 9.46^{\circ}$. Từ công thức \ref{eq:4.1} $\Rightarrow \delta_2 = 80.54^{\circ}$ \\
            \end{itemize}
        \subsection{Xác định số răng tương đương. Tính các hệ số $Y_{F1}$ và $Y_{F2}$ và so sánh độ bền uốn.}
            \begin{itemize}
                \item Số răng của bánh răng trụ răng thẳng tương đương được tính theo công thức 6.108 tài liệu tham khảo \cite{gtctm}:
                    \begin{equation}
                        z_{v} = \frac{z}{\cos{\delta_1}}
                        \label{eq:4.2}
                    \end{equation}
                    Từ công thức \ref{eq:4.2}:
                    \[
                    \Rightarrow
                    \begin{cases}
                        z_{v_1} = \frac{z_1}{\cos{\delta_1}} = \frac{20}{\cos{9.46}} = 20.28\\
                        z_{v_2} = \frac{z_2}{\cos{\delta_2}} = \frac{120}{\cos{80.54}} = 730.11
                    \end{cases}
                    \] 
                \item Chọn hệ số dịch chỉnh: chọn phương pháp dịch chỉnh là dịch chỉnh đều: $x_1 + x_2 = 0$ và hệ số dịch chỉnh bằng 0 $\Rightarrow x_1 = x_2 = 0$.      
                \item Hệ số $Y_{F1}$ và $Y_{F2}$ được tính theo công thức 6.80 tài liệu tham khảo \cite{gtctm}:                    
                    \begin{equation}
                        Y_{F} = 3.47 + \frac{13.2}{z_{v}} - \frac{27.9x}{z_{v}} + 0.092x^2
                        \label{eq:4.3}
                    \end{equation}
                    Từ công thức \ref{eq:4.3}:
                    \[
                    \Rightarrow
                    \begin{cases}
                        Y_{F1} = 3.47 + \frac{13.2}{25.35} = 4 \\
                        Y_{F2} = 3.47 + \frac{13.2}{912.64} = 3.49 
                    \end{cases}
                    \] 
                \item{So sánh độ bền uốn các bánh răng}
                    \begin{itemize}
                        \item Bánh dẫn: $\frac{[\sigma_{F1}]}{Y_{F1}} = \frac{360}{4} = 90$
                        \item Bánh bị dẫn: $\frac{[\sigma_{F2}]}{Y_{F2}} =\frac{329.14}{3.49} = 94.31$
                    \end{itemize}
                    $\Rightarrow$ Ta tính toán theo bánh dẫn có độ bền thấp hơn.
            \end{itemize}
        \subsection{Chọn chiều rộng vành khăn và hệ số xét đến ảnh hưởng sự phân bố tải trọng không đồng đều}
            \begin{itemize}
                \item Chọn hệ số chiều rộng vành khăn:$\Psi_{be} = 0.285$ và $\Psi_{bm} = 30$\\[0.3cm]
                $\Rightarrow$ Tỷ số $\frac{\Psi_{be} \cdot u}{2 - \Psi_{be}} = 1$. $\Psi_{bd} = \frac{\Psi_{bm}}{z_1} = \frac{30}{25} = 1.2$
                \item Giả sử bộ truyền được lắp trên ổ bi đỡ chặn. Từ bảng 6.18 tài liệu tham khảo \cite{gtctm}, ta chọn $K_{H\beta} = 1.34$.
                \item Hệ số xét đến ảnh hưởng sự phân bố tải trọng không đồng đều:
                \begin{equation*}
                    K_{F\beta} = 1 + (K_{H\beta} - 1) \cdot 1.5 = 1 + (1.34 - 1) \cdot 1.5 = 1.51.
                \end{equation*}
            \end{itemize}
        \subsection{Xác định môđun $m_e$ theo độ bền uốn}
            \begin{itemize}
                \item Xác định môđun chia trung bình $m_m$ theo công thức 6.119a tài liệu tham khảo \cite{gtctm}:
                \begin{equation}
                    m_m = 14\sqrt[3]{\frac{T_1 \cdot K_{F\beta} \cdot Y_{F_1}}{0.85\Psi_{bd} \cdot z_1^2 \cdot [\sigma_{F1}]}}
                    \label{eq:4.4}
                \end{equation}
                Từ công thức \ref{eq:4.4}:
                \begin{equation*}
                    m_m = 14\sqrt[3]{\frac{360.78 \cdot 1.51 \cdot 4}{0.85 \cdot 1.2 \cdot 25^2 \cdot 360}} = 2.96 \, \mathrm{mm}
                \end{equation*}
            \item Xác định môđun $m_e$ theo công thức 6.119b tài liệu tham khảo \cite{gtctm}
                \begin{equation}
                    m_e = \frac{m_m}{1 - 0.5 \cdot \Psi_{be}} = \frac{2.96}{1 - 0.5 \cdot 0.285} = 3.45 \, \mathrm{mm}
                    \label{eq:4.5}
                \end{equation}
                Theo tiêu chuẩn ta chọn môđun $m_e = 4 \, \mathrm{mm}$.
            \end{itemize}
    \section{XÁC ĐỊNH CÁC THÔNG SỐ HÌNH HỌC CHỦ YẾU CỦA BỘ TRUYỀN BÁNH RĂNG CÔN}
        \subsection{Xác định đường kính vòng chia}
            \begin{itemize}
                \item Đường kính vòng chia ngoài:
                    \begin{equation}
                        d_e = m_e \cdot z
                        \label{eq:4.6}
                    \end{equation}\
                    Từ công thức \ref{eq:4.6}:
                    \[
                    \Rightarrow
                    \begin{cases}
                        d_{e1} = 4 \cdot 25 = 100 \, \mathrm{mm} \\
                        d_{e2} = 4 \cdot 150 = 600 \, \mathrm{mm}
                    \end{cases}
                    \]
                \item Đường kính vòng chia trung bình:\\
                    \begin{equation}
                        d_m = d_e \cdot (1 - \psi_{be})
                        \label{eq:4.7}
                    \end{equation}\
                    Từ công thức \ref{eq:4.7}:
                    \[
                    \Rightarrow
                    \begin{cases}
                        d_{m1} = 100 \cdot (1 - 0.285) = 71.5 \, \mathrm{mm} \\
                        d_{m2} = 600 \cdot (1 - 0.285) = 429\, \mathrm{mm}
                    \end{cases}
                    \]
            \end{itemize}
        \subsection{Xác định chiều dài côn}
            \begin{itemize}
                \item Chiều dài côn ngoài:
                    \begin{equation}
                        R_e = 0.5m_e\sqrt{z_1^2 + z_2^2} = 0.5 \cdot 4 \cdot \sqrt{25^2 + 150^2} = 304.14 \, \mathrm{mm}
                        \label{eq:4.8}
                    \end{equation}
                \item Chiều dài côn trung bình:
                    \begin{equation}
                        R_m = 0.5m_m\sqrt{z_1^2 + z_2^2} = 0.5 \cdot 3.45 \cdot \sqrt{25^2 + 150^2} = 262.32 \, \mathrm{mm}
                        \label{eq:4.9}
                    \end{equation}
            \end{itemize}
        \subsection{Xác định chiều rộng vành khăn}
            \begin{equation}
                b = R_e \cdot \Psi_{be} = 304.14 \cdot 0.285 = 86.68 \, \mathrm{mm}
                \label{eq:4.10}
            \end{equation}
        \subsection{Xác định vận tốc vòng bánh răng}
            \begin{equation}
                v = \frac{\pi \cdot d_m \cdot n}{60000}
                \label{eq:4.11}
            \end{equation}
            \hspace*{0.6cm}Từ công thức \ref{eq:4.11}:
            \[
            \Rightarrow
            \begin{cases}
                v_1 = \frac{\pi \cdot 71.5 \cdot 72}{60000} = 0.269 \, \mathrm{m/s} \\
                v_2 = \frac{\pi \cdot 429 \cdot 12 }{60000} = 0.269 \, \mathrm{m/s}
            \end{cases}
            \]
            \hspace*{0.6cm}Theo bảng 6.3 tài liệu tham khảo \cite{gtctm} ta chọn cấp chính xác 9 với $v_{gh} = 2.5$ m/s.
    \section{XÁC ĐỊNH CÁC GIÁ TRỊ LỰC TÁC DỤNG LÊN BỘ TRUYỀN BÁNH RĂNG CÔN}
        \subsection{Lực trên bánh dẫn}
            \begin{itemize}
                \item Lực vòng:
                    \begin{equation}
                        F_{t1} = \frac{2T_1 \cdot 10^3}{d_{m1}} = \frac{2 \cdot 360.78 \cdot 10^3}{71.5} = 10091.75 \, \mathrm{N} \\
                        \label{eq:4.12}
                    \end{equation}
                \item Lực hướng tâm:
                    \begin{equation}
                        F_{r1} = F_{t1} \cdot \tan{\alpha} \cdot \cos{\delta_1} = 10091.75 \cdot \tan{20} \cdot \cos{9.46} = 3623.14  \, \mathrm{N}
                        \label{eq:4.13}
                    \end{equation}
                \item Lực dọc trục:
                    \begin{equation}
                        F_{a1} = F_{t1} \cdot \tan{\alpha} \cdot \sin{\delta_1} = 10091.75 \cdot \tan{20} \cdot \sin{11.32} = 603.71 \, \mathrm{N}
                        \label{eq:4.14}
                    \end{equation}
            \end{itemize}
        \subsection{Lực trên bánh bị dẫn}
            \begin{itemize}
                \item Lực vòng: 
                    \begin{equation*}
                        F_{t2} = F_{t1} = 10091.75 \, \mathrm{N} \\
                    \end{equation*}
                \item Lực hướng tâm:
                    \begin{equation*}
                        F_{r2} = F_{a1} = 603.71  \, \mathrm{N}
                    \end{equation*}
                \item Lực dọc trục:
                    \begin{equation*}
                        F_{a2} = F_{r1} = 3623.14 \, \mathrm{N}
                    \end{equation*}
            \end{itemize}
    \section{KIỂM NGHIỆM RĂNG THEO ĐỘ BỀN UỐN}
        \hspace*{0.6cm}Ứng suất uốn tại chân răng được tính theo công thức 6.118 tài liệu tham khảo \cite{gtctm}:
        \begin{equation}
            \sigma_{F} = \frac{Y_{F} \cdot F_{t} \cdot K_{F}}{0.85b_{w} \cdot m_{m}}
            \label{eq:4.15}
        \end{equation}
        \hspace*{0.6cm}Trong đó:
        \begin{itemize}
            \item Lực vòng trên bánh dẫn: $F_{t} = F_{t1} = 10091.75 \, \mathrm{N}$.
            \item Hệ số dạng răng tính theo số răng tương đương: $Y_{F} = Y_{F1} = 4$.
            \item Hệ số tải trọng: $K_{F} = K_{Fv} \cdot K_{F\beta} = 1.04 \cdot 1.51 = 1.57$. Trong đó: 
            \begin{itemize}
                \item $K_{F\beta} = 1.51$.
                \item Từ $v_{gh} = 2.5 m/s$  Theo bảng 6.18 ta chọn hệ số tải trọng động $K_{HV} = K_{FV} = 1.04$ (cấp chính xác 7).
            \end{itemize}
            \item Chiều rộng vành khăn: $b_{w} = 48\, \mathrm{mm}$.
            \item Môđun chia trung bình: $m_{m} = 3.45 \, \mathrm{mm}$.
        \end{itemize}
        $$\Rightarrow\sigma_{F1} = \frac{4 \cdot 10091.75 \cdot 1.57}{0.85 \cdot 86.68 \cdot 3.45} = 249.33 \, \mathrm{MPa} < [\sigma_{F1}] = 360 \, \mathrm{MPa}$$
        \hspace*{0.6cm}Do đó, điều kiện độ bền uốn được đảm bảo.
    \section{BẢNG THÔNG SỐ BỘ TRUYỀN BÁNH RĂNG CÔN}
        \begin{table}[H]
            \centering
            \begin{tabular}{|c|c|c|}
                \hline
                \textbf{Thông số} & \textbf{Bánh răng dẫn} & \textbf{Bánh răng bị dẫn} \\ \hline
                Tỉ số truyền & \multicolumn{2}{c|}{$u_{34} = 6$} \\ \hline
                Mômen xoắn (N.m) & \multicolumn{2}{c|}{$T_{III} = 360.78$} \\ \hline
                Số vòng quay (vg/ph) & \multicolumn{2}{c|}{$n_{III} = 72$} \\ \hline
                Góc nghiêng răng ($^o$) & \multicolumn{2}{c|}{$\beta = 0$} \\ \hline
                Chiều dài côn ngoài (mm) & \multicolumn{2}{c|}{$R_e = 304.14$} \\ \hline
                Chiều rộng vành khăn (mm) & \multicolumn{2}{c|}{$b_w = 86.68$} \\ \hline
                Chiều dài công trung bình (mm) & \multicolumn{2}{c|}{$R_m = 262.32$} \\ \hline
                Đường kính vòng chia ngoài (mm) & $d_{e1} = 100$ &  $d_{e2} = 600$ \\ \hline
                Góc côn chia ($^o$) & $\delta_1 = 9.46$ & $\delta_2 = 80.54$ \\ \hline
                Chiều cao răng ngoài (mm) & \multicolumn{2}{c|}{$h_e  = 8.8$} \\ \hline
                Chiều cao đầu răng ngoài (mm) & $h_{ae1}  = 4$ & $h_{ae2}  = 4$ \\ \hline
                Chiều cao chân răng ngoài (mm) & $h_{fe1}  = 4.8$ & $h_{fe2}  = 4.8  $ \\ \hline
                Đường kính đỉnh răng ngoài (mm) & $d_{ae1} = 107.89$ & $d_{ae2} = 601.31$ \\ \hline
                Chiều dày răng ngoài & (mm) $s_{e1} = 6.28$ & $s_{e2} = 6.28$ \\ \hline
                Góc chân răng ($^o$) & $\delta_{f1} = 0.904$ & $\delta_{f2} = 0.904$ \\ \hline
                Góc côn đỉnh ($^o$) & $\delta_ {a1} = 10.36$ & $\delta_ {a2} = 81.44$ \\ \hline
                Góc côn đáy ($^o$) & $\delta_{f1} = 8.56$ & $\delta_{f2} = 79.64$ \\ \hline
                Đường kính vòng chia trung bình (mm) & $d_{m1} = 71.5$ &  $d_{a2} = 429$ \\ \hline
                \makecell{Khoảng cách từ đỉnh côn \\ đến mặt phẳng vòng ngoài đỉnh răng (mm)} & $B_{1} = 299.35$ & $B_{2} = 46.04$ \\ \hline
                Module (mm) & \multicolumn{2}{c|}{$m_{te} = 4$} \\ \hline
                Module vòng trung bình (mm) & \multicolumn{2}{c|}{$m_{tm} = 2.96$} \\ \hline
                Module pháp trung bình (mm) & \multicolumn{2}{c|}{$m_{nm} = 3.45$} \\ \hline
                \makecell{Khoảng lệch tâm của \\ bánh răng côn tiếp tuyến (mm)} & \multicolumn{2}{c|}{$e = 0$} \\ \hline
                Số răng bánh răng (răng) & $z_1 = 25$ & $z_2 = 150$ \\ \hline
               
               
                Vận tốc vòng (m/s) &  \multicolumn{2}{c|}{$v = 0.269$} \\ \hline
            \end{tabular}
            \caption{Bảng thông số bộ truyền bánh răng côn}
        \end{table}
