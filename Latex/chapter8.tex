\chapter{BÔI TRƠN, DUNG SAI VÀ LẮP GHÉP}
    \section{BÔI TRƠN}
        \subsection{Bôi trơn trong hộp giảm tốc}
            \begin{itemize}
                \item Do các bộ truyền bánh răng trong hộp giảm tốc đều có $v > 12 m/s$ nên ta chọn phương pháp bôi trơn ngâm dầu. Lượng dầu bôi trơn thường vào khoảng $0.4 \div 0.8$ lít cho 1kW công suất truyền. Với vận tốc của bánh cấp nhanh $v = 5.03 m/s$, vật liệu thép C45, tra bảng 18-11 tài liệu tham khảo \cite{tltk2} ta được độ nhớt 80/11 ứng với 500C. 
                \item Theo bảng 18-13 tài liệu tham khảo \cite{tltk2}, ta chọn loại dầu bôi trơn là dầu Ôt máy kéo AK20. 
            \end{itemize}
        \subsection{Lắp bánh răng lên trục và điều chỉnh sự ăn khớp}
            \begin{itemize}
                \item Đối với bánh răng côn, việc điều chỉnh được tiến hành trên cả hai bánh răng dẫn và bị dẫn.
                \item Dịch chuyển trục cùng với các bánh răng đã cố định trên nó nhờ bộ đệm điều chỉnh có chiều dày khác nhau lắp giữa nắp ổ và vỏ hộp. Việc điều chỉnh như thế này khá thuận tiện. Dịch chuyển các bánh răng trên trục đã cố định, sau đó định vị lần lượt từng bánh một. Việc điều chỉnh này khá phức tạp.
                \item Lưu ý: Độ điều chỉnh phải đạt tối thiểu là $70\%$ trên bề mặt răng.
            \end{itemize}
    \section{CHỌN CẤP CHÍNH XÁC}
        \begin{itemize}
            \item Đối với hệ thống trục, chọn cấp chính xác là 6. Vì gia công lỗ phức tạp hơn gia công trục, do đó chọn cấp chính xác gia công lỗ thấp hơn trục 1 cấp, ta chọn cấp 7.
            \item Đối với bánh răng, chọn cấp chính xác là 9 như đã tính toán.
        \end{itemize}
    \section{CHỌN KIỂU LẮP}
        \subsection{Bánh răng bị dẫn}
            \hspace*{0.6cm}Chọn kiểu lắp H7/k6 do mối ghép không yêu cầu tháo lắp thường xuyên, bánh răng lắp trên trục chịu tải trọng tĩnh, vừa, va đập nhẹ.
        \subsection{Ổ lăn}
            \hspace*{0.6cm}Vòng trong ổ lăn chịu tải tuần hoàn, ta lắp ghép theo hệ thống trục lắp trung gian để vòng ổ không trượt trên bề mặt trục khi làm việc. Do đó, ta phải chọn mối lắp k6, lắp trung gian có độ dôi, tạo điều kiện mòn đều ổ (trong quá trình làm việc nó sẽ quay làm mòn đều). \\
            \hspace*{0.6cm}Vòng ngoài của ổ lăn không quay nên chịu tải cục bộ, ta lắp theo hệ thống lỗ. Để ổ có thể di chuển dọc trục khi nhiệt đô tăng trong quá trình làm việc, ta chọn kiểu lắp trung gian H7.
        \subsection{Then}
            \hspace*{0.6cm}Trong mối ghép then, kích thước lắp ghép là bề rộng b của then. Đối với trục, ta chọn kiểu lắp trung gian N9/H9, đối với bạc ta chọn kiểu lắp trung gian Js9/h9.
        \subsection{Vòng chắn dầu}
            \hspace*{0.6cm}Vì vòng chắn dầu cần quay theo trục để dầu bên ngoài không bắn vào trong ổ lăn và dễ dang tháo lắp nên ta chọn lắp trung gian H7/js6.
        \subsection{Vòng phớt và nắp ổ}
            \hspace*{0.6cm}Để mối ghép cố định khi làm việc nhưng các chi tiết dễ dàng dịch chuyển với nhau khi điều chỉnh nên ta chọn kiểu lắp hở H7/h6 cho nắp ổ lăn và H8/e8 cho vòng phớt. 
    \section{BẢNG DUNG SAI}
        \subsection{Bảng dung sai lắp ghép bánh răng}
            \begin{table}[H]
                \centering
                \begin{tabular}{|c|c|c|c|c|c|c|}
                    \hline
                    \textbf{Mối lắp} & \textbf{Kích thước} & \textbf{Kiểu lắp} & $\mathbf{es (\mu m)}$ & $\mathbf{ei (\mu m)}$ & $\mathbf{ES (\mu m)}$ & $\mathbf{EI (\mu m)}$\\
                    \hline
                    Bánh đai & $\phi 22$ & H7/k6 & 15 & 2 & 21 & 0 \\
                    \hline
                    Bánh răng bị dẫn & $\phi 60$ & H7/k6 & 21 & 2 & 30 & 0 \\
                    \hline
                    Bánh răng côn dẫn & $\phi 40$ & H7/k6 & 18 & 2 & 25 & 0 \\
                    \hline
                \end{tabular}
                \caption{Bảng dung sai lắp ghép bánh răng}
            \end{table}
        \subsection{Bảng dung sai lắp ghép then}
            \begin{table}[H]
                \centering
                \begin{tabular}{|c|c|c|c|c|}
                    \hline
                    \multirow{3}{*}{\begin{tabular}[c]{@{}c@{}}Kích thước tiết\\diện then\\$b \times h$\end{tabular}} &
                    \multicolumn{2}{c|}{\begin{tabular}[c]{@{}c@{}}Sai lệch giới hạn\\chiều rộng rãnh then\end{tabular}} &
                    \multicolumn{2}{c|}{Chiều sâu rãnh then} \\
                    \cline{2-5}
                    &\begin{tabular}[c]{@{}c@{}}Trên\\trục\end{tabular} &
                    \begin{tabular}[c]{@{}c@{}}Trên\\bạc\end{tabular} &
                    \multirow{2}{*}{\begin{tabular}[c]{@{}c@{}}Sai lệch giới hạn\\trên trục $t_1$\end{tabular}} &
                    \multirow{2}{*}{\begin{tabular}[c]{@{}c@{}}Sai lệch giới hạn\\trên bạc $t_2$\end{tabular}} \\
                    \cline{2-3}
                    & N9 & Js9 & & \\
                    \hline
                    $6 \times 6$ &
                    \begin{tabular}[c]{@{}c@{}}0\\-0.036\end{tabular} &
                    \begin{tabular}[c]{@{}c@{ }}+0.018\\-0.018\end{tabular} &
                    0.2 & 0.2 \\
                    \hline
                    $18 \times 11$ &
                    \begin{tabular}[c]{@{}c@{}}0\\-0.043\end{tabular} &
                    \begin{tabular}[c]{@{}c@{}}+0.0215\\-0.0215\end{tabular} &
                    0.2 & 0.2 \\
                    \hline
                    $12 \times 8$ &
                    \begin{tabular}[c]{@{}c@{}}0\\-0.043\end{tabular} &
                    \begin{tabular}[c]{@{}c@{}}+0.0215\\-0.0215\end{tabular} &
                    0.2 & 0.2 \\
                    \hline
                \end{tabular}
                \caption{Bảng dung sai lắp ghép then}
            \end{table}
    \subsection{Bảng dung sai lắp ghép ổ lăn}
        \begin{table}[H]
            \centering
            \begin{tabular}{|c|c|c|c|c|c|}
                \hline
                Mối lắp & Kiểu lắp & es ($\mu m$) & ei ($\mu m$) & ES ($\mu m$) & EI ($\mu m$) \\
                \hline
                Vòng trong ổ trục II & $\varnothing30k6$ & +18 & +2 & & \\
                \hline
                Vòng trong ổ trục III & $\varnothing50k6$ & +18 & +2 & & \\
                \hline
                Vòng ngoài ổ trục II & $\varnothing62H7$ & & & +30 & 0 \\
                \hline
                Vòng ngoài ổ trục III & $\varnothing110H7$ & & & +35 & 0 \\
                \hline
            \end{tabular}
            \caption{Bảng dung sai lắp ghép ổ lăn}
        \end{table}
    \subsection{Bảng dung sai các chi tiết khác}
        \begin{table}[htbp]
            \centering
            \begin{tabular}{|c|c|c|c|c|c|c|}
                \hline
                Mối lắp & Kích thước & Kiểu lắp & es ($\mu m$) & ei ($\mu m$) & ES ($\mu m$) & EI ($\mu m$) \\
                \hline
                Nắp ổ trục II & $\varnothing62$ & H7/h6 & 0 & -19 & +30 & 0 \\
                \hline
                Nắp ổ trục III & $\varnothing110$ & H7/h6 & 0 & -22 & +35 & 0 \\
                \hline
                Vòng chặn đầu trục II & $\varnothing35$ & H7/js6 & +8 & -8 & +25 & 0 \\
                \hline
                Vòng chặn đầu trục III & $\varnothing55$ & H7/js6 & +9,5 & -9,5 & +30 & 0 \\
                \hline
                Vòng phớt trục II & $\varnothing25$ & H8/e8 & -40 & -73 & +33 & 0 \\
                \hline
                Vòng phớt trục III & $\varnothing45$ & H8/e8 & -50 & -89 & +39 & 0 \\
                \hline
            \end{tabular}
            \caption{Bảng dung sai các chi tiết khác}
        \end{table}

            
            